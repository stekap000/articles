\documentclass{article}

% Language setting
% Replace `english' with e.g. `spanish' to change the document language
\usepackage[english]{babel}

% Set page size and margins
% Replace `letterpaper' with `a4paper' for UK/EU standard size
\usepackage[a4paper,top=2cm,bottom=2cm,left=3cm,right=3cm,marginparwidth=1.75cm]{geometry}

% Useful packages
\usepackage{caption}
\usepackage{graphicx}
\usepackage{amsmath}
\usepackage{amsfonts}
\usepackage{amssymb}
\usepackage{amsbsy}
\usepackage{bbm}
\usepackage{mathabx}
\usepackage[colorlinks=true, allcolors=blue]{hyperref}

\title{Some notions of Measure Theory}
\author{Stefan Kapetanović}

\begin{document}
\maketitle

\begin{abstract}
This is a collection of some measure theoretic notions without much rigor. Main idea behind this is to make it easier to intuitively understand reasons for the measure theoretic approaches in path tracing.
\end{abstract}

\section{Measure}
Finding formulas for volumes of various different objects (like a sphere), which were often represented as sets, led to a natural question of whether it is possible to find a formula for the volume of any set that is a subset of an arbitrary dimensional Euclidean space. Because of this, there was an attempt to construct a mapping $\mu:\mathbb{R}^k\to\mathbb{R}_{0}^{+}$ that would take some arbitrary subset of $\mathbb{R}^k$ and assign a positive number to it. This consideration of arbitrary subsets eventually led to the Banach-Tarski paradox, which represents an "impossible" construction where we consider unit ball $B\in\mathbb{R}^3$ that is divided into disjoint subsets. Then, these subsets can be isometrically mapped such that we end up with 2 unit balls. This is a completely valid construction, but it doesn't align with our notion of volume, since it should be invariant under isometries (just moving a thing should not change its volume). Finally, this led to attempts to characterize subsets to which we can assign volume (or n-dimensional volume), first for Euclidean spaces, and then for other, more complicated ones.

\subsection{Notion of Measure}
Measure tries to generalize notions like length, area, count, etc. and to put them in a more formal context, by trying to assign value to a certain set. As an example, $[a, b]$ range is a set, and we can try to assign a value to it, somehow. One straightforward approach is to simply assign an intuitive notion of length to this range. We do this by defining a function that takes a set and outputs a number.

\[ L([a, b]) = b - a \]

We can now think of some properties that this length has in order to try and abstract properties of measure. Lets imagine a line of some length. First, we measure one smaller portion of this line. This portion is just a set of points that we will denote as $S_1$ and its length is $L(S_1)$. Now, we measure another portion $S_2$ that does not overlap $S_1$ and get a length of $L(S_2)$. Intuitively, we know that if we join these two pieces together, their total length should just be the sum of individual lengths. This is one property that we found.

\[ L(S_1 \cup S_2) = L(S_1) + L(S_2),\;given\;that\; S_1 \cap S_2 = \varnothing \]

Another thing that we can intuitively conclude about length is the following. We observe part of the line $S_1$ with length $L(S_1)$. Now, we measure one smaller portion of that part $S_2$ and get the length $L(S_2)$. If we remove that smaller part, we intuitively know that the length of the remaining portion is the difference of lengths of larger and smaller part.

\[ L(S_1 \backslash S_2) = L(S_1) - L(S_2),\;given\;that\; S_1 \cap S_2 = S_2 \]

We can also partially overlap two pieces $S_1$ and $S_2$ and ask what is the length of their intersection. To find this, we can add two individual lengths, but in that case, we have counted the length of intersection twice, so we also need to subtract it once.

\[ L(S_1 \cup S_2) = L(S_1) + L(S_2) - L(S_1 \cap S_2) \]

From this, we derive the length of intersection.

\[ L(S_1 \cap S_2) = L(S_1) + L(S_2) - L(S_1 \cup S_2) \]

There is also a question of what would be the length of "empty" line ie. empty set. Intuitively, we would expect this to be 0.

\[ L(\varnothing) = 0 \]

From all of these things, we can try to define what an abstract notion of measure means, by saying that if the length $L$ is a measure and $S_1$ and $S_2$ are measurable, meaning we can assign lengths $L(S_1)$ and $L(S_2)$ to them, then the following things are also measurable:

\begin{itemize}
    \item $S_1 \cup S_2$
    \item $S_1 \backslash S_2$
    \item $S_1 \cap S_2$
    \item $\varnothing$
\end{itemize}

Notice that we can extend a measure of $S_1 \cup S_2$, where $S_1$ and $S_2$ are disjoint, to the measure of a union of arbitrary number of countable sets (if we have them), and not just those two. We can also calculate this measure from the measures of individual sets just like we did when we had only two sets (same thing holds for intersections and differences).

\[ L(\bigcup\limits_{i=1}^{k} S_i) = L(S_1) + L(S_2) + \dots = \sum\limits_{i=1}^{k}L(S_i),\;where\;k \in \mathbb{N} \]

Now that we extracted these properties, we can try to express them with a smaller number of rules, before moving onto experimenting with them and seeing if they are useful. This is not strictly necessary, but it allows us to be more clear about what kind of system we are dealing with. To move to a smaller set of rules, we will need some relations from set theory. Suppose we have some set and two of its subsets $A_1$ and $A_2$. The following relations hold.

\[ A_1 \backslash A_2 = A_1 \cap A_2^c \]
\[ A_1 \cap A_2 = (A_1^c \cup A_2^c)^c \]

Looking at these relations and at the things that we concluded should be measurable, we can see that if we add a rule that tells us that complement of a measurable set must be measurable, then we don't need a rule for difference, since it is covered by intersection and complement rules. But, we don't even need intersection rule since it is covered by union and complement rules. So we are left with rules for union, complement and empty set.
Of course, it is arbitrary which set of rules we actually pick, as long as it is minimal and can generate remaining rules.

\subsection{Ground rules}

All the previous properties are usually expressed with two groups of rules. First group contains rules that describe the domain of a measure function (length in previous section). They tell us what is measurable. Second group contains rules that describe measure function itself.

To define what is measurable ie. which sets can be mapped to numbers using measure function, we first define a set $G$ that serves as a playground, and then we sample subsets from it, forming another set that contains these subsets. This choice of sets is arbitrary as long as the rules that define what is measurable hold. This new set of subsets is called $\sigma$-algebra if it obeys all the rules, and it defines what can be measured. If we denote a set of subsets of $G$ by $S$, then it is $\sigma$-algebra if the following is satisfied:

\begin{itemize}
    \item $\varnothing \in S$ - We must always be able to assign value to empty set, which is zero, just like with the length example, so the empty set must always be part of $\sigma$-algebra.
    \item $A \in S \implies A^c \in S,\;A^c = G \backslash A$ - If we can measure one part of G (think about line), then we can measure remaining part of G. Additionally, since we know from the first rule that an empty set must be in $\sigma$-algebra, then based on the current rule we know that its complement must also be in there ie. $\varnothing^c = G \in S$. This tells us that the space we are working with must itself be measurable.
    \item $S_i \in S \implies \bigcup S_i \in S$ - Countable union of measurable subsets of $G$ is also measurable. Intuitively, this tells us that we can combine pieces that we already measured in order to measure some other part.
\end{itemize}

Now that we know what it even means for something to be measurable, we can define properties of measure function that will map measurable sets to numbers. If we denote this function by $\mu$ and $\sigma$-algebra by $S$, then the following holds:

\begin{itemize}
    \item $\mu: S \rightarrow [0, \infty]$ - It assigns non-negative number to a measurable set. This rule comes naturally when thinking about length or size of things in the usual sense. There is a variation of Measure Theory where measure function assigns any real number to a measurable set.
    \item $\mu(\varnothing) = 0$ - Just like with length, empty set always has a measure 0.
    \item $\mu(\bigcup S_i) = \sum\mu(S_i)\;given\;that\; S_i \in S,\;S_i \cap S_j = \varnothing\:(i \neq j)$ - If the measurable sets are disjoint, then the measure of their union is the sum of their individual measures.
\end{itemize}

Given all these rules we can define a Measure Space that groups them all together. If we denote global set we are working with as $G$, $\sigma$-algebra as $S$, and measure function as $\mu$, then a Measure Space $M$ is given by a triplet:

\[ M = (G, S, \mu)\]

\subsection{Null set and measure space completion}
If we have a measure $\mu$, then a null set $N$, with respect to that measure, is a set with measure 0 ie.

\[ \mu(N) = 0 \]

We know that empty set is always a null set. An example of a null set that is not an empty set is the following. Lets imagine a real line and focus on rational numbers. We will try to cover all the rational numbers using sticks. We start with a stick of size $1$ and place it onto the first rational number. This can be done because they are countable ie. it is possible to find a bijection $\mathbb{Q}\to\mathbb{N}$. Our first stick will cover more than one rational number, but we don't care as long as the first one is covered. Next, we cover second rational number with stick of size $\frac{1}{2}$, third one with the stick of size $\frac{1}{4}$ etc. We do this for all rational numbers, and we end up with the total length $L$ for all sticks.

\[ L=\sum_{i=0}^{\infty}\frac{1}{2^i}=2\]

This means that we can cover all rational numbers with length 2. But, nothing stops us from using smaller sticks. For example, first stick can have the size of $\frac{1}{10000}$, second $\frac{1}{10000^2}$ etc. By looking at the limit, where stick sizes approach 0, the total length also approaches zero.

\[ \lim_{x \to 0}\sum_{i=1}^{\infty}\frac{1}{x^i}=\lim_{x \to 0}(\sum_{i=0}^{\infty}\frac{1}{x^i}-1)=\lim_{x \to 0} (\frac{1}{1-x} - 1)=0 \]

After all, it seems that the length of all rational numbers is just 0. This way, we have found that $\mu(\mathbb{Q})=0$ for length measure $\mu$ and that $\mathbb{Q}$ is a null set under that measure.
\newline\indent Similarly, we can try to find what would be the measure of the line within $\mathbb{R}^2$, given standard measure for rectangle area $\mu$. Lets take a straight strip of plane, with width $w$, and place it over this line, such that line is in the middle. Now, we partition this strip into rectangles that have width $w$, just like the strip, and length $l$. Every rectangle now covers portion of the line with length $l$. All that is left to do now, is to shrink rectangles. As we do this, they become line segments with no area. If $p$ is the given line, $S$ chosen strip, $R_i$ rectangles, and $\mu$ standard measure for rectangle area, we have

\[\mu(p)=\lim_{w \to 0}\mu(S)=\lim_{w \to 0}\sum\mu(R_i)=\lim_{w \to 0}\sum lw=0\]

Lets take countable number of lines in $\mathbb{R}^2$. Then, we can use the same argument, just with multiple strips, to show that the measure of their union is 0 (we can ignore their intersections, since they are countable within the line).

\[ \mu(\bigcup p_i) = \lim_{w \to 0}\mu(\bigcup S_i) = \lim_{w \to 0}\sum\mu(S_i) = 0 \]

These are additional examples of null sets, now in $\mathbb{R}^2$ and with a different measure. One important thing to note from these examples is that it seems that the measure of some measurable subset is always zero when that subset consists of countable number of objects of lower dimension than the dimension that measure $\mu$ operates on. In the first example, a countable set of points embedded within the line has no length. In the second example, line and countable union of lines do not have area. 
\newline\indent Apart from demonstrating that there are null sets other than empty set, these examples indicate that it is intuitive to say that any subset of a null set has a measure 0. The problem is that it can happen that a null set has a subset that is not measurable. As an example, lets construct one such case. Lets take a set $G=\{a,b,c\}$, with $\sigma$-algebra $S=\{\varnothing, \{a\},\{b,c\},\{a,b,c\}\}$ and a measure $\mu$ defined as

\[\mu=\begin{cases}
    \varnothing & \to 0 \\
    \{a\} & \to 1 \\
    \{b,c\} & \to 0 \\
    \{a,b,c\} & \to 1
\end{cases}\]

Now, since $\{b,c\}$ has a measure 0, it is a null set, but its subsets $\{b\}$ and $\{c\}$ are not measurable, since they are not in $\sigma$-algebra S. Since, after all, we would like measures to behave intuitively (like lengths), it makes sense to try to define that every subset of a null set has a measure 0.

\[ \mu(B)=0,\;A \subseteq B \implies \mu(A)=0 \]

If we define this, it allows us to create a new measure space where we don't have to worry about null space subsets that are not measurable. For this purpose, we introduce a concept of a measure space completion, which creates larger measure space. We will work with some measure space $M=(G,S,\mu)$. First, we collect all subsets of null sets into one set $N$.

\[ N=\{X:X \subseteq B,\;B \in S,\;\mu(B)=0\} \]

Now, we will expand measure space $M$ in the sense that we will make more things measurable by expanding $\sigma$-algebra $S$. To do this, we will use property of measure of unions. Since a measure of a union of sets is a sum of measures of those sets, we know that the following must hold

\[ \mu(A)=0,\;D \in S \implies \mu(D \cup A)=\mu(D)+\mu(A)=\mu(D) \]

This means that we can use previously constructed set $N$ and combine all of its elements with the elements of $\sigma$-algebra S, thus creating larger $\sigma$-algebra without changing a measure of previously measurable sets. Lets denote larger $\sigma$-algebra with $S_{L}$.

\[ S_{L}=\{D \cup A: D \in S,\;A \in N\} \]

Now, the new measure $\mu_{L}$ of this larger measure space can stay the same for all previous values, but also define values for new sets in $\sigma$-algebra.

\[ \mu_{L}(D \cup A)=\mu(D) + \mu(A) = \mu(D) \]

This way, we have constructed a new measure space $M_{L}$ that is called the completion of the measure space $M$.

\[ M_{L}=(G,S_{L},\mu_{L}) \]

The whole point of doing this is to, in a sense, have a space to work with, without worrying about whatever is contained within its null sets. As an example, lets try to expand a measure space that we used in example for the existence of some subsets of null sets that are not measurable. We worked with this measure space

\[ G=\{1,2,3\},\;S=\{\varnothing, \{a\},\{b,c\},\{a,b,c\}\},\; \mu=\begin{cases}
    \varnothing & \to 0 \\
    \{a\} & \to 1 \\
    \{b,c\} & \to 0 \\
    \{a,b,c\} & \to 1
\end{cases} \]

In this case, there is only one null set $\{b,c\}$, apart from empty set. Therefore, a set containing all subsets of null sets is

\[N=\{\varnothing, \{a\}, \{b\}, \{a,b\}\}\]

Combining this with $\sigma$-algebra $S$ using union, we get

\[ S_{L}=\{\} \]

\subsection{Examples of measures}

Before moving on, here are some examples of measures.

\subsubsection{Borel measure on $\mathbb{R}^n$}
This is a measure that is defined on the so called Borel algebra, which is a specific $\sigma$-algebra (note that the Borel algebra is more general, but we only consider Euclidean spaces). Before its definition, lets consider what it represents in 2-dimensional Euclidean space. There, it tries to say that any rectangle in the plane is measurable and assigns area to it. To represent arbitrary rectange, we will use Cartesian product of two ranges $[a_1, b_1]\times[a_2, b_2]$, which just gives us a new set that contains all possible pairs of points with first coordinate from the first range, and second coordinate from the second range. Now, we just say that all such rectangles are measurable by putting them into $\sigma$-algebra. Additionally, we naturally assign area to them by constructing measure $\mu$, such that
\[\mu([a_1, b_1]\times[a_2, b_2])=(b_1-a_1)(b_2-a_2)\]
which is just a standard rectangle area. Using this, we can combine rectangles in the plane to approximate some weird shape. Then, we can find area of this shape by summing areas of rectangles, because we know it is measurable since it is a union of measurable rectangles. In order to have $\sigma$-algebra, apart from rectangles, we also need to add all possible unions of rectangles, and complements of all elements, since they are also measurable.
\newline\indent To extend this to $\mathbb{R}^n$, we just need a way to represent n-dimensional rectangles and their volume. We represent rectangles using Cartesian products and their volume using the product of lengths of their sides. Represented with formulas, this is n-dimensional rectangle with its measure:

\[ [a_1,b_1]\times[a_2,b_2]\times\dots\times[a_n,b_n] = \prod_{i=1}^{n}[a_i,b_i] \]
\[ \mu(\prod_{i=1}^{n}[a_i,b_i])=(b_1-a_1)(b_2-a_2)\dots(b_n-a_n)=\prod_{i=1}^{n}(b_i-a_i) \]

\section{Lebesgue integral}
Lebesgue integration is one of the main notions that will be introduced because it is required for integration of more exotic functions and it also offers a nice conceptual framework for defining things like radiance in a more measure theoretic way.

\subsection{Initial motivation}
Reason for introducing a new concept of integration is to cover another set of mathematical objects and to define what it means to integrate them. Looking at a standard Riemann integral, we can find functions that are not integrable. As an example, let's define a sequence of functions:

\[ S_n(x) = \begin{cases}
    1 & x = \frac{p}{q},\: p \in \mathbb{N}_0,\: q \in \mathbb{N},\: q \leq n \\
    0 & otherwise
\end{cases},\: x \in [0, 1] \]

Every function in this sequence is Riemann integrable if we know that $n$ is finite. For example, for $n=2$, we know that $q$ can only be 1 or 2. Since $x \in [0, 1]$ must hold, we know that $p$ can only take values that will leave fraction $\frac{p}{q}$ in $[0, 1]$ range. These are possible cases:

\[\frac{0}{1},\;\frac{1}{1},\;\frac{0}{2},\;\frac{1}{2},\;\frac{2}{2}\]

From this, we can see that we only have three distinct fractions ie. values for $x$ for which the function $S_2(x)$ has a value 1. This would be the definition for second function in the sequence:

\[S_2(x) = \begin{cases}
    1 & x \in \{0,\frac{1}{2},1\} \\
    0 & otherwise
\end{cases},\: x \in [0, 1] \]

This function is continuous everywhere, except in 3 points, because it has value 0 everywhere and value 1 only in those 3 points, so we can find its Riemann integral because the number of discontinuities is finite. Using the same reasoning for any function in this sequence for finite $n$, we can see that the number of points in which those functions are discontinuous is finite, meaning that they also have Riemann integral. As we go to higher values of $n$, every next function has a value of 1 for more and more fractions ie. rational numbers. As $n$ goes to infinity, $q$ starts taking any natural number as value, and in this process $\frac{p}{q}$ starts representing any rational number in $[0, 1]$ range.

\[ \lim_{n\to\infty}S_n(x) = \begin{cases}
    1 & x \in \mathbb{Q}_{0}^{+} \\
    0 & otherwise
\end{cases} = \begin{cases}
    1 & x \in \mathbb{Q} \\
    0 & otherwise
\end{cases},\; x \in [0, 1] \]

This is Dirichlet function, which is often represented as an indicator function $\mathbbm{1}_{\mathbb{Q}}(x)$ for the set of rational numbers, meaning it has a value 1 for any element in that set. Since it represents the limit of the previous sequence, this function has infinite number of discontinuities (every rational number in range[0, 1]). If we try to find Riemann integral by upper and lower approximations, one would equal 0 and the other 1, meaning that there is no well defined single value, so the integral does not exist. On the other hand, we know that $\mathbb{Q}$ is countable and $\mathbb{R}\backslash\mathbb{Q}$ is uncountable, so intuitively, irrational numbers should dictate the final value, which is 0. The problem is that Riemann integration can't help us answer this. In order to find a solution, we need a new kind of integral. To be able to define it, we need a previously introduced notion of measure.

\end{document}