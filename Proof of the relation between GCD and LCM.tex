\documentclass{article}

\usepackage[english]{babel}
\usepackage[a4paper,top=2cm,bottom=2cm,left=3cm,right=3cm,marginparwidth=1.75cm]{geometry}
\usepackage{caption}
\usepackage{graphicx}
\usepackage{amsmath}
\usepackage{amsfonts}
\usepackage{amssymb}
\usepackage{amsbsy}
\usepackage{bbm}
\usepackage{mathabx}
\usepackage[colorlinks=true, allcolors=blue]{hyperref}

\title{Proof of the relation between GCD and LCM}
\author{ Stefan Kapetanović }
\date{March 2025}

\begin{document}
\maketitle

\begin{abstract}
This is a proof of relation $a \cdot b = GCD(a, b) \cdot LCM(a, b)$ which directly relies only on divisibility of natural numbers and doesn't use any existing result that is not, in itself, immediately obvious.
\end{abstract}

\section{Proof}

We need to prove that $a \cdot b = GCD(a, b) \cdot LCM(a, b)$.
Given two numbers $a, b \in \mathbb{N}$, and using $GCD(a, b) = k$, the following relations always hold

\[ \exists! \, k1 \in \mathbb{N} \; , \; a = k1 \cdot k \]
\[ \exists! \, k2 \in \mathbb{N} \; , \; b = k2 \cdot k \]

Combining these relations, we get that

\[ a \cdot b = k1 \cdot k2 \cdot k \cdot k = k1 \cdot k2 \cdot k \cdot GCD(a, b) \]

In order to prove the original relation, we need to show that $k1 \cdot k2 \cdot k$ is the minimal number divisible by both $a$ and $b$. Divisibility is easy to show

\[ a \divides k1 \cdot k2 \cdot k = k2 \cdot a \]
\[ b \divides k1 \cdot k2 \cdot k = k1 \cdot b \]

We must show that there does not exist some natural number $L$ that is divisible by both $a$ and $b$, and smaller than $k1 \cdot k2 \cdot k$. We will assume that a number with such properties exists

\[ \exists \, L \in \mathbb{N} \]
\[ a \divides L \iff \exists! \, l1 \in \mathbb{N} , \; L = l1 \cdot a = l1 \cdot k1 \cdot k \]
\[ b \divides L \iff \exists! \, l2 \in \mathbb{N}, \; L = l2 \cdot b = l2 \cdot k2 \cdot k \]
\[ L < k1 \cdot k2 \cdot k \]

Firstly, using previous relations, we can find the order between $k1$ and $l2$, and also between $k2$ and $l1$.

\[ l1 \cdot k1 \cdot k < k1 \cdot k2 \cdot k \iff l1 < k2 \]
\[ l2 \cdot k2 \cdot k < k1 \cdot k2 \cdot k \iff l2 < k1 \]

Secondly, we can find a relation between all $k1$, $k2$, $l1$ and $l2$.

\[ l1 \cdot k1 \cdot k = l2 \cdot k2 \cdot k \iff l1 \cdot k1 = l2 \cdot k2 \iff \frac{l1}{l2} \cdot k1 = k2 \]

Since $k2$ is a natural number, the left side of the previous relation must also be a natural number. Now we show that $\frac{l1}{l2}$ can't be a natural number. If we assume that it is, then

\[ \exists! \, p \in \mathbb{N} , \; \frac{l1}{l2} = p \iff l1 = p \cdot l2 \]

If we now use this result instead of $l1$ in the relation for $L$, we get

\[ L = l1 \cdot a = p \cdot l2 \cdot a \]

Combining this with relation for $L$ that involves $b$, we get

\[ p \cdot l2 \cdot a = l2 \cdot b \iff p \cdot a = b \]

If this was true, then we would have $GCD(a, b) = GCD(a, p \cdot a) = a$, instead of $GCD(a, b) = k$. Therefore, our assumption was wrong and we know that $\frac{l1}{l2}$ can't be a natural number. Since the left side must be a natural number, this implies that $k1$ must be a multiple of $l2$

\[ \exists! \, g \in \mathbb{N}, \; k1 = g \cdot l2 \]

If we use this instead of $k1$ in equation for $L$, and combine it with another equation for $L$, we get

\[ l1 \cdot g \cdot l2 \cdot k = l2 \cdot k2 \cdot k \iff k2 = g \cdot l1 \]

We know that $a = k1 \cdot k$ and $b = k2 \cdot k$, and by using new expressions for $k1$ and $k2$, we get

\[ a = g \cdot l2 \cdot k \]
\[ b = g \cdot l1 \cdot k \]

Since $GCD(a, b) = k$, this is only possible if $g = 1$, because we would otherwise have $GCD(a, b) = g \cdot k$. This gives us $k1 = l2$ and $k2 = l1$. This is in contradiction with the order constraints between them that are the consequence of assumption regarding existence of $L$, namely

\[ l1 < k2 \]
\[ l2 < k1 \]

This shows that the assumption was wrong and that the number $L$ does not exist, and therefore, we know that $k1 \cdot k2 \cdot k$ is the minimal number that is divisible by both $a$ and $b$. This means that it is equal to LCM(a, b). From this, it follows that

\[ a \cdot b = GCD(a, b) \cdot LCM(a, b) \]

\rightline{$\blacksquare$}

\end{document}
