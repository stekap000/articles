\documentclass{article}

\usepackage[english]{babel}
\usepackage[a4paper,top=2cm,bottom=2cm,left=3cm,right=3cm,marginparwidth=1.75cm]{geometry}
\usepackage{caption}
\usepackage{graphicx}
\usepackage{amsmath}
\usepackage{amsfonts}
\usepackage{amssymb}
\usepackage{amsbsy}
\usepackage{bbm}
\usepackage{mathabx}
\usepackage[colorlinks=true, allcolors=blue]{hyperref}

\title{Proof of the relation between GCD and LCM}
\author{ Stefan Kapetanović }
\date{March 2025}

\begin{document}
\maketitle

\begin{abstract}
This is a proof of relation $a \cdot b = GCD(a, b) \cdot LCM(a, b)$ which directly relies only on divisibility of natural numbers and doesn't use any existing result that is not, in itself, immediately obvious.
\end{abstract}

\section{Proof}

We need to prove that $a \cdot b = GCD(a, b) \cdot LCM(a, b)$.
Given two numbers $a, b \in \mathbb{N}$, the following relations always hold

\[ \exists! \, k_1 \in \mathbb{N} \; , \; a = k_1 \cdot GCD(a, b) \]
\[ \exists! \, k_2 \in \mathbb{N} \; , \; b = k_2 \cdot GCD(a, b) \]

We will first handle trivial cases where at least one of the numbers $a$ or $b$ is equal to $GCD(a, b)$.

\[ a = GCD(a, b) \iff b = k_2 \cdot a \iff b = LCM(a, b) \iff a \cdot b = GCD(a, b) \cdot LCM(a ,b) \]
\[ b = GCD(a, b) \iff a = k_1 \cdot b \iff a = LCM(a, b) \iff a \cdot b = GCD(a, b) \cdot LCM(a ,b) \]
\[ a = b = GCD(a, b) \iff a = b = LCM(a, b) \iff a \cdot b = GCD(a, b) \cdot LCM(a ,b) \]
\[\]

Now we will prove that the original relation also holds in non-trivial case ie. when neither $a$ nor $b$ is equal to $GCD(a, b)$. We will use shorthand $GCD(a, b) = k$. From expressions for $a$ and $b$, we get

\[ a \cdot b = k_1 \cdot k_2 \cdot k \cdot k = k_1 \cdot k_2 \cdot k \cdot GCD(a, b) \]
\[\]

In order to prove the original relation, we need to show that $k_1 \cdot k_2 \cdot k$ is the minimal number divisible by both $a$ and $b$. Divisibility is easy to show

\[ a \divides k_1 \cdot k_2 \cdot k = k_2 \cdot a \]
\[ b \divides k_1 \cdot k2 \cdot k = k_1 \cdot b \]

We must show that there does not exist some natural number $L$ that is divisible by both $a$ and $b$, and smaller than $k_1 \cdot k_2 \cdot k$. We will assume that a number with such properties exists

\[ \exists \, L \in \mathbb{N} \]
\[ a \divides L \iff \exists! \, l_1 \in \mathbb{N} , \; L = l_1 \cdot a = l1 \cdot k_1 \cdot k \]
\[ b \divides L \iff \exists! \, l_2 \in \mathbb{N}, \; L = l_2 \cdot b = l_2 \cdot k_2 \cdot k \]
\[ L < k_1 \cdot k_2 \cdot k \]

Firstly, using previous relations, we can find the order between $k_1$ and $l_2$, and also between $k_2$ and $l_1$.

\[ l_1 \cdot k_1 \cdot k < k_1 \cdot k_2 \cdot k \iff l_1 < k_2 \]
\[ l_2 \cdot k_2 \cdot k < k_1 \cdot k_2 \cdot k \iff l_2 < k_1 \]

Secondly, we can find a relation between all $k_1$, $k_2$, $l_1$ and $l_2$.

\[ l_1 \cdot k_1 \cdot k = l_2 \cdot k_2 \cdot k \iff l_1 \cdot k_1 = l_2 \cdot k_2 \iff \frac{l_1}{l_2} \cdot k_1 = k_2 \]

Since $k_2$ is a natural number, the left side of the previous relation must also be a natural number. Now we show that $\frac{l_1}{l_2}$ can't be a natural number. If we assume that it is, then

\[ \exists! \, p \in \mathbb{N} , \; \frac{l_1}{l_2} = p \iff l_1 = p \cdot l_2 \]

If we now use this result instead of $l_1$ in the relation for $L$, we get

\[ L = l_1 \cdot a = p \cdot l_2 \cdot a \]

Combining this with relation for $L$ that involves $b$, we get

\[ p \cdot l_2 \cdot a = l_2 \cdot b \iff p \cdot a = b \]

If this was true, then we would have $GCD(a, b) = GCD(a, p \cdot a) = a$, instead of $GCD(a, b) = k$. Therefore, our assumption was wrong and we know that $\frac{l_1}{l_2}$ can't be a natural number. Since the left side must be a natural number, this implies that $k_1$ must be a multiple of $l_2$

\[ \exists! \, g \in \mathbb{N}, \; k_1 = g \cdot l_2 \]

If we use this instead of $k_1$ in equation for $L$, and combine it with another equation for $L$, we get

\[ l_1 \cdot g \cdot l_2 \cdot k = l_2 \cdot k_2 \cdot k \iff k_2 = g \cdot l_1 \]

We know that $a = k_1 \cdot k$ and $b = k_2 \cdot k$, and by using new expressions for $k_1$ and $k_2$, we get

\[ a = g \cdot l_2 \cdot k \]
\[ b = g \cdot l_1 \cdot k \]

Since $GCD(a, b) = k$, this is only possible if $g = 1$, because we would otherwise have $GCD(a, b) = g \cdot k$. This gives us $k_1 = l_2$ and $k_2 = l_1$. This is in contradiction with the order constraints between them that are the consequence of assumption regarding existence of $L$, namely

\[ l_1 < k_2 \]
\[ l_2 < k_1 \]

This shows that the assumption was wrong and that the number $L$ does not exist, and therefore, we know that $k1 \cdot k_2 \cdot k$ is the minimal number that is divisible by both $a$ and $b$. This means that it is equal to LCM(a, b). From this, it follows that

\[ a \cdot b = GCD(a, b) \cdot LCM(a, b) \]

\rightline{$\blacksquare$}

\end{document}
